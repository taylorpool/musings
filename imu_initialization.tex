\documentclass{article}

\usepackage{amsmath}
\usepackage{amsfonts}
\usepackage{IEEEtrantools}

\title{Attitude and IMU Bias Initialization}
\author{Taylor Pool}
\date{Mar 7, 2024}

\begin{document}

\maketitle

\section{Problem Statement}
Let $w$ be the inertial frame.
Let $r$ be the robot body frame aligned with the IMU. 
Let $^w n$ be the normal force in the world frame.

Given IMU measurements $({^r \omega^k}, {^r a^k})$, we seek to estimate
\begin{itemize}
	\item Gyroscope bias $b_\omega$
	\item Accelerometer bias $b_a$
	\item Rotation of robot in world frame ${^w R_r}$
\end{itemize}

\section{Gyroscope}

We assume that the following relationship holds
\begin{equation}
	{^r \omega^k} = {^r \tilde{\omega}^k} + {b_\omega} + {\eta_\omega^k}
\end{equation}

Where the bias is normally distributed as $^r_\omega b \sim \mathcal{N}(0, \Sigma_\omega)$.
Each white noise term $\eta_\omega^k$ is a sample of $\eta_\omega \sim \mathcal{N}(0,\Sigma_{\eta_\omega})$
Next, we assume that in this case, the robot is stationary, so that ${^r \tilde{\omega}^k} = 0$.

Now, we have
\begin{IEEEeqnarray}{rCl}
	{^r \omega^k} & = & {^r \tilde{\omega}^k} + {b_\omega} + {\eta_\omega^k} \\
	{^r \omega^k} & = & {b_\omega} + {\eta_\omega^k} \\
\end{IEEEeqnarray}

Manipulating the equation yields for each sample while the robot is stationary:
\begin{IEEEeqnarray}{c}
	{b_\omega} = {^r \omega^k} - {\eta_\omega^k}
\end{IEEEeqnarray}

Computing the expectation of the righthand side gives
\begin{IEEEeqnarray}{c}
	\label{eqn:bias_gyro}
	{b_\omega} = \mathbb{E}\left[ {^r \omega} \right]
\end{IEEEeqnarray}

Thus we have an expression for the bias of the gyroscope in Equation \ref{eqn:bias_gyro}.

\section{Rotation and Accelerometer Bias}

We seek to find ${^r R_w}, b_a$, where no yaw is present (does not make sense for an initialization).

Now, we have the following model for the accelerometer:
\begin{IEEEeqnarray}{c}
	{^r a^k} = {^r \tilde{a}^k} - {^r R_w^k} {^w g} + b_a + \eta_a^k
\end{IEEEeqnarray}

Then we again assume that the robot is stationary:
\begin{IEEEeqnarray}{c}
	{^r a^k} = - {^r R_w^k} {^w g} + b_a + \eta_a^k
\end{IEEEeqnarray}

Thinking about the no-yaw constraint, this means that we need the gravity vector to live in the x-z plane.
So this means that we first need to project the nominal world gravity vector onto the xz plane.

In order to do this, we first assume that the 

We first seek to solve the following problem

Given $a^ x, b^ x \in \mathbb{R}^3$, find $^a R_b \in \mathrm{SO}(3)$ such that $^a x = ^a R_b * ^b x$, with no yaw.

Then we can think about this as the following:

\end{document}
