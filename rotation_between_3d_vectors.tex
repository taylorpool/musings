\documentclass{article}
\usepackage{amsmath}
\usepackage{amsfonts}
\usepackage{amssymb}
\usepackage{IEEEtrantools}
\usepackage{hyperref}

\newcommand{\transpose}[1]{#1^\mathrm{T}}

\title{Rotation Between Two Vectors in $\mathbb{R}^3$}
\author{Taylor Pool}

\begin{document}
\maketitle

\section{Problem Statement}

Let $a, b \in \mathbb{R}^3$ be unit vectors.
Find a rotation ${^a R_b}$ such that
\begin{IEEEeqnarray}{lCl}
    b & = & {^b R_a} \odot a
\end{IEEEeqnarray}

\section{Solution}
\label{solution}

First, compute the inner product between $a$ and $b$:
\begin{IEEEeqnarray}{lCl}
    \cos{\theta} & = & a \cdot b
\end{IEEEeqnarray}

\subsection{Non-opposite Case}
\label{non-opposite-case}

Now, if for some $\epsilon > 0$,
\begin{IEEEeqnarray}{lCl}
    \cos{\theta} & >= & \epsilon - 1
\end{IEEEeqnarray}
Then this means that the two vectors are sufficiently non-opposite.
We can then compute the cross product from $a$ to $b$ as
\begin{IEEEeqnarray}{lCl}
    \omega \sin{\theta} & = & a \times b
\end{IEEEeqnarray}
Now, from the double angle identities, we have
\begin{IEEEeqnarray}{lCl}
    \label{sqrt-eqn}
    \cos{\frac{\theta}{2}} & = & \sqrt{\frac{\cos{\theta} + 1}{2}} \\
    \omega \sin{\frac{\theta}{2}} & = & \omega \sin{\theta} \frac{1}{2 \cos{\frac{\theta}{2}}}
\end{IEEEeqnarray}
Then we may express ${^b R_a}$ as a quaternion:
\begin{IEEEeqnarray}{lCl}
    {^b R_a} & = & \begin{bmatrix}
        \cos{\frac{\theta}{2}} \\
        \omega \sin{\frac{\theta}{2}}
    \end{bmatrix}
\end{IEEEeqnarray}

\subsection{Opposite Case}
Suppose now that for $\epsilon > 0$,
\begin{IEEEeqnarray}{lCl}
    \cos{\theta} & < & \epsilon - 1
\end{IEEEeqnarray}

In this case, $a$ and $b$ are pointing in opposite directions, which has direct numerical stability issues,
since the cross product between two parallel vectors is the zero vector.
To solve this issue, we compute an intermediate rotation $^c R_a$ of $\pi$ radians about an axis perpendicular to $a$.

\subsubsection{Finding the axis}
We select the index, $i$ of $a$ with the minimum absolute value.
\begin{IEEEeqnarray}{lCl}
    i & = & \arg \min_j | a_j |
\end{IEEEeqnarray}
Then we compute our axis of rotation, $\mu$ as
\begin{IEEEeqnarray}{lCl}
    \mu & = & \mathrm{normalized}(e_i - a a_i)
\end{IEEEeqnarray}
where $e_i$ is the unit vector with 1 as the $i$-th element.

\subsubsection{Defining the Intermediate Rotation}
Then we may define $^c R_a$ in terms of a quaternion as:
\begin{IEEEeqnarray}{lCl}
    {^c R_a} & = & \begin{bmatrix}
        0 \\
        \mu
    \end{bmatrix}
\end{IEEEeqnarray}

\subsubsection{Final Solution}
Now, let $c = {^c R_a} \odot a$, and we may find ${^b R_c}$ via the method proposed in Section \ref{non-opposite-case}.
Finally
\begin{IEEEeqnarray}{lCl}
    {^b R_a} & = & {^b R_c} \circ {^c R_a}
\end{IEEEeqnarray}

\section{Remarks}

\subsection{Efficiency}
The solution presented Section \ref{solution} is very efficient.
Even though $\sin$ and $\cos$ terms appear, these quantities are calculated via
the inner and cross products.
The most expensive operation of the procedure is the square root in Equation \ref{sqrt-eqn}.

\subsection{Inspiration}

We take inspiration from Ethan Eade's approach, however Eade does not specify how to choose the axis of intermediate rotation, which
is a nontrivial task.
Furthermore, Eade's approach is in terms of rotation matrices. For more information, see \url{https://ethaneade.com/rot_between_vectors.pdf}.

\end{document}
