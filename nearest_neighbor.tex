\documentclass{article}

\usepackage{biblatex}

\addbibresource{resources.bib}

\title{Nearest Neighbor for Robotics}

\author{Taylor Pool}

\begin{document}
\maketitle

In robotics, nearest neighbor search is fundamentally important for procedures such as point cloud registration.
However, in contrast with other fields, robotics needs the search index to update as quick as possible to ensure real time performance.

There are two main approaches to doing so.

\begin{enumerate}
\item K-D Tree: This structure divides at a binary level all the way down
\item Octree: This structure recursively subdivides into 8 children until a small enough group remains (which is stored in a list)
\item Voxel Grid: This structure is not really great because it can introduce discretization errors.
\end{enumerate}

The issue with all of the above approaches is that they generally don't perform well in the incremental addition and deletion space.
To this end, there have been papers such as the incremental kd tree \cite{cai_ikd-tree_2021} and the incremental octree \cite{zhu_i-octree_2024} that seek to rectify this issue.
Both are probably satisfactory for the given problem space and should probably be used over other methods.

\printbibliography

\end{document}
