\documentclass{article}
\usepackage[utf8]{inputenc}
\usepackage{amsmath}
\usepackage{amsfonts}
\usepackage{amssymb}
\usepackage{algpseudocodex}
\usepackage{bm}
\usepackage{mathtools}
\usepackage{IEEEtrantools}

\newcommand{\transpose}[1]{#1^\mathrm{T}}
\newcommand{\skewsym}[1]{\lfloor #1 \rfloor_\times}
\newcommand{\norm}[1]{|| #1 ||}
\newcommand{\sign}[1]{{\mathrm{sign}\left( #1 \right)}}

\title{Quaternion Notes}
\author{Taylor Pool}
\date{December 2022}

\begin{document}

\maketitle

\section{Introduction}

Quaternions are an important representation of orientation in robotics. They are defined as

Note that quaternions are vectors $\mathbb{R}^4$. They just have a special group operation overlayed on top of the vector space.

Unit quaternions on the other hand are a \pmb{manifold} embedded in $\mathbb{R}^4$.

\begin{align*}
	q & = q_w + q_x i + q_y j + q_z k
\end{align*}
Or we also say a quaternion is
\begin{align*}
	\pmb{q} & = \begin{bmatrix}
		            q_w \\
		            q_x \\
		            q_y \\
		            q_z
	            \end{bmatrix}        \\
	q       & = \begin{bmatrix}
		            1 & i & j & k
	            \end{bmatrix} \pmb{q}
\end{align*}

Where
\begin{align*}
	ij  & = k              \\
	jk  & = i              \\
	i^2 & = j^2 = k^2 = -1 \\
\end{align*}
We can then derive
\begin{align*}
	ik & = -j
\end{align*}

A special group is the group of unit quaternions such that
\begin{align*}
	|| \pmb{q} || & = || \pmb{q} ||^2 = q_w^2 + q_x^2 + q_y^2 + q_z^2 = 1
\end{align*}

Suppose we have a quaternion $\pmb{q} \ne \pmb{0}$. We wish to find $\Delta \pmb{q}$ such that $\pmb{q} + \Delta \pmb{q} $ is a unit quaternion, while minimizing $||\Delta \pmb{q}||^2$

Then we may formulate this as a convex optimization problem as follows:

\begin{align}
	\min \quad          & || \Delta \pmb{q} ||^2                  \\
	\mathrm{s.t.} \quad & || \pmb{q} + \Delta \pmb{q} ||^2 -1 = 0
\end{align}

We introduce the augmented lagragian with multiplier $\lambda$
\begin{align*}
	\mathcal{L} & = || \Delta \pmb{q} ||^2 + \lambda \left[ || \pmb{q} + \Delta \pmb{q} ||^2 - 1 \right]                                                                                                                                            \\
	            & = \transpose{\left(\Delta \pmb{q} \right)} \Delta \pmb{q} + \lambda \left[ \transpose{\left(\pmb{q} + \Delta \pmb{q}\right)} \left(\pmb{q} + \Delta \pmb{q} \right) - 1 \right]                                                   \\
	            & = \transpose{\left(\Delta \pmb{q} \right)} \Delta \pmb{q} + \lambda \left[ \transpose{\pmb{q}} \pmb{q} + 2\transpose{\left(\Delta \pmb{q} \right)} \pmb{q} + \transpose{\left(\Delta \pmb{q} \right)} \Delta \pmb{q}  - 1 \right] \\
\end{align*}
Then we have
\begin{align*}
	\mathrm{D}\mathcal{L} & = 2 \transpose{\left(\Delta \pmb{q} \right)} + 2 \lambda \left[ \transpose{\pmb{q}} + \transpose{\left(\Delta \pmb{q} \right)} \right] \\
	                      & = \transpose{\pmb{0}}
\end{align*}
Thus,
\begin{align*}
	\Delta \pmb{q} + \lambda \left[ \pmb{q} + \Delta \pmb{q} \right] & = \pmb{0}                              \\
	\Delta \pmb{q} (1 + \lambda) + \lambda \pmb{q}                   & = \pmb{0}                              \\
	\Delta \pmb{q}                                                   & = -\frac{\lambda \pmb{q}}{1 + \lambda}
\end{align*}
Then by substitution we have
\begin{align*}
	|| \pmb{q} + \Delta \pmb{q} ||^2 -1 = 0                         \\
	|| \pmb{q} - \frac{\lambda \pmb{q}}{1 + \lambda} ||^2 - 1 & = 0 \\
	|| \frac{1}{1 + \lambda} \pmb{q} ||^2 - 1                 & = 0 \\
	|| \pmb{q} ||^2 - (1+\lambda)^2                           & = 0 \\
\end{align*}
We can formulate this as a quadratic in $\lambda$ as follows
\begin{align*}
	\lambda^2 + 2\lambda + 1 - || \pmb{q} ||^2 & = 0                                                          \\
	\lambda                                    & = \frac{-2 \pm \sqrt{4 - 4\left(1- ||\pmb{q}||^2\right)}}{2} \\
	                                           & = -1 \pm || \pmb{q} ||
\end{align*}
So we have
\begin{align*}
	\Delta \pmb{q} & = - \frac{\lambda}{1 + \lambda} \pmb{q}                           \\
	               & = - \frac{-1 \pm || \pmb{q} ||}{1 + -1 \pm || \pmb{q} ||} \pmb{q} \\
	               & = \frac{1 \mp || \pmb{q} || }{\pm || \pmb{q} ||} \pmb{q}
\end{align*}
These are two potential solutions but we seek the one that minimizes $|| \Delta \pmb{q} || ^2$.
\begin{align*}
	|| \Delta \pmb{q} ||^2 & = \left( \frac{1 \mp || \pmb{q} ||}{\pm || \pmb{q} ||} \right)^2 || \pmb{q} ||^2
\end{align*}
Since $|| \pmb{q} || > 0$, we have that
\begin{align*}
	\left( \frac{1 - ||\pmb{q}|| }{||\pmb{q}||} \right)^2 & = \frac{1 - 2 || \pmb{q} || + ||\pmb{q}||^2}{||\pmb{q}||^2}                                           \\
	                                                      & < \frac{1 - 2 || \pmb{q} || + ||\pmb{q}||^2}{||\pmb{q}||^2} + \frac{4 || \pmb{q} ||}{|| \pmb{q} ||^2} \\
	                                                      & = \frac{1 + 2 || \pmb{q} || + ||\pmb{q}||^2}{||\pmb{q}||^2}                                           \\
	                                                      & = \left( \frac{1 + ||\pmb{q}|| }{-||\pmb{q}||} \right)^2
\end{align*}
Thus, we have eliminated one of the proposals and have arrived at our final solution:
\begin{align}
	\Delta \pmb{q} & = \frac{1 - ||\pmb{q} ||}{|| \pmb{q} ||} \pmb{q}
\end{align}
Then $||\Delta \pmb{q}||^2$ is minimized while satisfying $|| \pmb{q} + \Delta \pmb{q} || = 1$
Then
\begin{align*}
	\pmb{q} + \Delta \pmb{q} & = \frac{1 - ||\pmb{q}||}{||\pmb{q}||} \pmb{q} + \pmb{q} \\
	                         & = \frac{\pmb{q}}{||\pmb{q}||}
\end{align*}

\section{Optimization over Quaternions}

\begin{align*}
	\prescript{w}{}{\pmb{q}}_{a} & = \begin{bmatrix}
		                                 \prescript{w}{}{s}_a \\
		                                 \prescript{w}{}{v}_a
	                                 \end{bmatrix} \\
	\prescript{a}{}{\pmb{q}}_b   & = \begin{bmatrix}
		                                 \prescript{a}{}{s}_b \\
		                                 \prescript{a}{}{v}_b
	                                 \end{bmatrix}
\end{align*}

\subsection{Quaternion Composition}

\begin{align*}
	L \left( \pmb{q} \right) & = \begin{bmatrix}
		                             q_w & -q_x & -q_y & -q_z \\
		                             q_x & q_w  & -q_z & q_y  \\
		                             q_y & q_z  & q_w  & -q_x \\
		                             q_z & -q_y & q_x  & q_w
	                             \end{bmatrix} \\
\end{align*}

\begin{align*}
	R \left( \pmb{q} \right) & = \begin{bmatrix}
		                             q_w & -q_x & -q_y & -q_z \\
		                             q_x & q_w  & q_z  & -q_y \\
		                             q_y & -q_z & q_w  & q_x  \\
		                             q_z & q_y  & -q_x & q_w
	                             \end{bmatrix}
\end{align*}

\begin{align*}
	\prescript{w}{}{\pmb{q}}_b & = f(\prescript{w}{}{\pmb{q}}_a, \prescript{a}{}{\pmb{q}}_b)              \\
	                           & = L \left( \prescript{w}{}{\pmb{q}}_a \right) \prescript{a}{}{\pmb{q}}_b \\
	                           & = R \left( \prescript{a}{}{\pmb{q}}_b \right) \prescript{w}{}{\pmb{q}}_a \\
\end{align*}

\begin{align*}
	Df(\pmb{q}_1, \pmb{q}_2) & = \begin{bmatrix}
		                             R(\pmb{q}_2) & L(\pmb{q}_1)
	                             \end{bmatrix}
\end{align*}

\subsection{Quaternion-Point Rotation}

\begin{align*}
	M\left( \pmb{q} \right) & = \begin{bmatrix}
		                            q_w^2 + q_x^2 - q_y^2 - q_z^2 & 2(q_x q_y - q_w q_z)          & 2(q_x q_z + q_w q_y)          \\
		                            2(q_x q_y + q_w q_z)          & q_w^2 - q_x^2 + q_y^2 - q_z^2 & 2(q_y q_z - q_w q_x)          \\
		                            2(q_x q_z - q_w q_y)          & 2(q_y q_z + q_w q_x)          & q_w^2 - q_x^2 - q_y^2 + q_z^2
	                            \end{bmatrix}
\end{align*}

\begin{align*}
	N(\pmb{q}, x) & = \begin{bmatrix}
		                  2 q_w x - 2 q_z y + 2 q_y z  & 2 q_x x + 2 q_y y + 2 q_z z & -2 q_y x + 2 q_x y + 2 q_w z  & -2 q_z x - 2 q_w y + 2 q_x z \\
		                  2 q_z x + 2 q_w y - 2 q_x z  & 2 q_y x - 2 q_x y - 2 q_w z & 2 q_x x + 2 q_y y + 2 q_z z   & 2 q_w x - 2 q_z y + 2 q_y z  \\
		                  -2 q_y x + 2 q_x y + 2 q_w z & 2 q_z x + 2 q_w y - 2 q_x z & - 2 q_w x + 2 q_z y - 2 q_y z & 2 q_x x + 2 q_y y + 2 q_z z  \\
	                  \end{bmatrix}
\end{align*}

\begin{align*}
	\prescript{b}{}{x} & = g\left(\prescript{b}{}{\pmb{q}}_a, \prescript{a}{}{x}\right)                              \\
	                   & = \left(\prescript{b}{}{\pmb{q}}_a * \hat{\prescript{a}{}{x}}\right)^\vee                   \\
	                   & = M\left( \prescript{b}{}{\pmb{q}}_a \right) \prescript{a}{}{x}                             \\
	                   & = N\left( \prescript{b}{}{\pmb{q}}_a, \prescript{a}{}{x} \right) \prescript{b}{}{\pmb{q}}_a
\end{align*}

\begin{align*}
	Dg(\pmb{q}, x) & = \begin{bmatrix}
		                   N(\pmb{q}, x) & M(\pmb{q})
	                   \end{bmatrix}
\end{align*}

\subsection{Pose Composition}

\begin{align*}
	\prescript{c}{}{T}_b & = \left( \prescript{c}{}{\pmb{q}}_b, \prescript{c}{}{t}_b \right)                                                                                                                \\
	\prescript{b}{}{T}_a & = \left( \prescript{b}{}{\pmb{q}}_a, \prescript{b}{}{t}_a \right)                                                                                                                \\
	\prescript{c}{}{T}_a & = h\left( \prescript{c}{}{T}_b, \prescript{b}{}{T}_a \right)                                                                                                                     \\
	                     & = \left( f\left(\prescript{c}{}{\pmb{q}}_b, \prescript{b}{}{\pmb{q}}_a \right), g \left( \prescript{c}{}{\pmb{q}}_b, \prescript{b}{}{t}_a \right) + \prescript{c}{}{t}_b \right) \\
\end{align*}

Then we have

\begin{align*}
	Dh\left(\prescript{c}{}{T}_b, \prescript{b}{}{T}_a \right) & = \begin{bmatrix}
		                                                               D_1 f\left(\prescript{c}{}{\pmb{q}}_b, \prescript{b}{}{\pmb{q}}_a \right) & 0   & D_1 f\left(\prescript{c}{}{\pmb{q}}_b, \prescript{b}{}{\pmb{q}}_a \right) & 0                                                                   \\
		                                                               D_1 g\left( \prescript{c}{}{\pmb{q}}_b, \prescript{b}{}{t}_a\right)       & I_3 & 0                                                                         & D_2 g\left( \prescript{c}{}{\pmb{q}}_b, \prescript{b}{}{t}_a\right)
	                                                               \end{bmatrix} \\
	                                                           & = \begin{bmatrix}
		                                                               R\left( \prescript{c}{}{\pmb{q}}_b \right)                       & 0_{4 \times 3} & L\left( \prescript{b}{}{\pmb{q}}_a \right) & 0_{4 \times 3}                             \\
		                                                               N\left( \prescript{c}{}{\pmb{q}}_b, \prescript{b}{}{t}_a \right) & I_3            & 0_{3 \times 4}                             & M\left( \prescript{c}{}{\pmb{q}}_b \right)
	                                                               \end{bmatrix}
\end{align*}

\section{Unit Quaternion Algebra}

\begin{IEEEeqnarray}{lCl}
	L(q) & = & \begin{bmatrix}
		s & -\transpose{v}   \\
		v & sI + \skewsym{v}
	\end{bmatrix}
\end{IEEEeqnarray}

\begin{IEEEeqnarray}{lCl}
	R(q) & = & \begin{bmatrix}
		s & -\transpose{v}   \\
		v & sI - \skewsym{v}
	\end{bmatrix}
\end{IEEEeqnarray}

\begin{IEEEeqnarray}{lCl}
	L(q^{-1}) & = & \transpose{L}(q) \\
	R(q^{-1}) & = & \transpose{R}(q)
\end{IEEEeqnarray}

\begin{IEEEeqnarray}{lCl}
	q_3 & = & q_1 \odot q_2 \\
	& = & L(q_1) q_2 \\
	& = & R(q_2) q_1
\end{IEEEeqnarray}

\section{Exponential Map}

Let $\vec{\omega} = \omega \theta$. Where $\omega$ is a unit vector in $\mathbb{R}^3$.

\begin{IEEEeqnarray}{lCl}
	\mathrm{Exp}(\vec{\omega}) & = & \begin{bmatrix}
		\cos{\frac{\theta}{2}} \\
		\omega \sin{\frac{\theta}{2}}
	\end{bmatrix}
\end{IEEEeqnarray}

To compute numerically given $\vec{\omega}$, first compute $\theta = \norm{\vec{\omega}}$.
Then if $|\theta| < \epsilon$, then by L'Hopital's rule, we have
\begin{IEEEeqnarray}{lCl}
	\mathrm{Exp}(\vec{\omega}) & = & \mathrm{normalized} \begin{bmatrix}
		\cos{\frac{\theta}{2}} \\
		\frac{\vec{\omega}}{2}
	\end{bmatrix}
\end{IEEEeqnarray}
Otherwise if $|\theta| \ge \epsilon$, then we have
\begin{IEEEeqnarray}{lCl}
	\mathrm{Exp}(\vec{\omega}) & = & \begin{bmatrix}
		\cos{\frac{\theta}{2}} \\
		\frac{\sin{\frac{\theta}{2}}}{\theta}\vec{\omega}
	\end{bmatrix}
\end{IEEEeqnarray}

\section{Logarithmic Map}

Note that
\begin{IEEEeqnarray}{lCl}
	q & = & \begin{bmatrix}
		\cos{\frac{\theta}{2}} \\
		\omega \sin{\frac{\theta}{2}}
	\end{bmatrix} \\
	& = & \begin{bmatrix}
		s \\
		v
	\end{bmatrix}
\end{IEEEeqnarray}
We seek to find $\vec{\omega} = \omega \theta$.

By Taylor's Remainder Theorem,
\begin{IEEEeqnarray}{lCl}
	| \theta | < \delta & \Rightarrow & \left| \sin{\theta} - \theta \right| < \frac{1}{3!} \delta^3 \\
	& \Rightarrow & \left| \cos{\theta} - \left(1 - \frac{1}{2} \theta^2 \right) \right| < \frac{1}{4!} \delta^4
\end{IEEEeqnarray}

Let the max error between $\sin{\theta}$ and $\theta$ be given by $\epsilon$.
\begin{IEEEeqnarray}{lCl}
	\epsilon & = & \frac{1}{3!} \delta^3 \\
	\delta & = & (6 \epsilon)^{1/3}
\end{IEEEeqnarray}

We define the sign function as
\begin{IEEEeqnarray}{lCl}
	\sign{x} & = & \begin{cases}
		1  & x \ge 0 \\
		-1 & x < 0
	\end{cases} \\
	& = & 2 (x \ge 0) - 1
\end{IEEEeqnarray}

Note that
\begin{IEEEeqnarray}{lCl}
	|s - \frac{\theta}{2} | < \frac{1}{3!} (\frac{\delta}{2})^3
\end{IEEEeqnarray}


Then we may compute the logarithm as follows:

\begin{algorithmic}
	\State Choose max error $\epsilon > 0$
	\State Compute $\delta = 2(6 \epsilon)^{1/3}$
	\State Compute $s_{\epsilon} = \cos{\delta}$
	\If{$s > s_{\epsilon}$}
	\State \Return $2*v$
	\Else
	\State $\theta \gets 2 \arccos(s)$
	\State \Return $\frac{v}{\norm{v}} \sign{s} \theta$
	\EndIf
\end{algorithmic}




\end{document}
